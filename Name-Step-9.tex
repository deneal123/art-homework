\documentclass[12pt]{article}
\usepackage[utf8]{inputenc}
\usepackage[T1]{fontenc}
\usepackage{amsmath,amsfonts,amssymb}
\usepackage{graphicx}
\usepackage{a4wide}
\usepackage{hyperref}

\begin{document}

\paragraph{Title:} The title of your research scientific paper

\paragraph{Abstract:} You already have written this one. Try to improve it. Include a description of your problem, its motivation  and goals. An optimization-style problem statement is welcome.

\paragraph{Datasets:}  A brief description of data in the computational experiment and. Links to the datasets. The datasets shall be open-source. The data shall be ready-to-model.
\begin{enumerate}
\item Dataset title, basic to start and link~\cite{potanin2019genetic}.
\item Dataset title, main or alternative and link~\cite{potanin2019genetic}.
\item Dataset title, synthetic. 
\end{enumerate}

\paragraph{References:}  Papers with a fast intro and the basic solution to compare.
\begin{enumerate}
\item The formulation of the problem~\cite{potanin2019genetic}.
\item A baseline and new results~\cite{potanin2019genetic}.
\item Some fast introduction~\cite{potanin2019genetic}.
\end{enumerate}

\paragraph{Basic solution:} A link~\cite{potanin2019genetic} to the code of the baseline algorithm. It shows the state of the art and will be compared with the proposed solution.

\paragraph{Authors:} Expert and Consultant

\paragraph{Supplementary:}  A link to detailed problem statement and materials is welcome.

\bibliographystyle{unsrt}
\bibliography{Name-theArt}
\end{document}