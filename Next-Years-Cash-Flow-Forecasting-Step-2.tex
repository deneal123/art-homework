\documentclass[12pt]{article}
\usepackage[utf8]{inputenc}
\usepackage[T1]{fontenc}
\usepackage{amsmath,amsfonts,amssymb}
\usepackage{graphicx}
\usepackage{a4wide}
\title{~~~~~~~ Industrial project description \newline «Next Year's Cash-Flow Forecasting»}
%\author{not specified}
\date{}
\begin{document}
\maketitle

%\begin{abstract}
Answer the question to outline your project. Choose one of the roles: an {Expert} or an~\textbf{Analyst}.
%\end{abstract}
% \paragraph{Keywords:} The Art On Scientific Research, Abstract Reconstruction, Please Put Yours 


\section{Planning the industrial research project}
% Before planning the research, the analyst and (\textbf{expert}) discuss the key issues. After the long dash~--- our remarks.

\begin{enumerate}
\item Goal of the project. (\textbf{Expected development result.})~---
The goal of this project is to develop a reliable forecasting model to forecast next year's cash flow for a large retail company. The forecast will assist the finance department in planning budgets, managing liquidity and identifying investment opportunities for the coming year.
\item Applied problem solved in the project. (\textbf{How will the result be used?})~--- The result of this forecast will be directly used by the finance team to make informed budgeting and strategic planning decisions. This will allow for better cash flow management, ensuring sufficient liquidity to cover operating expenses, as well as identifying excess cash for potential investments or debt management.
\item Description of historical measured data. (\textbf{Formats and timing.})~--- Historical data includes monthly cash flow statements for the last five years, broken down by revenue, expenses, taxes and operating expenses. Data is stored in CSV and Excel formats, where each row represents one month's worth of data. In addition, external factors such as market trends, inflation rates and sector performance indicators are taken into account.
\item Quality criteria. (\textbf{How is the quality of the obtained result measured, what is in the report?})~--- The quality of the result obtained will be measured by the accuracy of cash flow forecasts. Metrics such as root mean square error (RMSE) and mean absolute percentage error (MAPE) will be optimized to provide reliable forecasts. Cross-validation techniques will be used to evaluate model performance on unseen data.
\item Project feasibility. (\textbf{How to show that the project is feasible, list of possible risks.})~--- To demonstrate the feasibility of the project, error analysis will be performed throughout the development process. Potential risks include readjusting past trends that may not apply to future market conditions or mishandling external factors. Feasibility can be demonstrated by continuously monitoring model accuracy on the validation set and performing sensitivity analyzes on external data.
\item Conditions necessary for successful project implementation. (\textbf{Organization of work.}) ~--- Successful project implementation requires a well-organized, high-quality data set that includes a variety of internal financial indicators, external economic indicators, as well as an understanding of seasonality in business. Collaboration with financial experts is essential to ensure domain-specific knowledge is included in the model.
\item Solution methods. (\textbf{Procedure libraries.})~--- The core methodology will include time series forecasting methods, particularly using models such as ARIMA, Prophet or LSTM-based deep learning models for sequential data. Various hypotheses will be tested, including the impact of macroeconomic factors on cash flow trends and identification of seasonal patterns.
\end{enumerate}

\section{Research or development?}
In other words, novelty or technological advancement?


{Analyst:} The impact of this research will enhance a company's financial forecasting capabilities by improving the accuracy of cash flow forecasts. This will improve decision making in budgeting, operational management and strategic financial planning.

{Expert:} This model is expected to last about 1-2 years before requiring an update to accommodate market changes and new financial models. In the future, this model will likely be replaced by more advanced models incorporating real-time data feeds and dynamic market adjustments, which will improve real-time forecasting accuracy.

%\bibliographystyle{unsrt}
%\bibliography{Name-theArt}
\end{document}